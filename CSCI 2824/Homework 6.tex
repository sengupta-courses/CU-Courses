
\documentclass[11pt]{amsart}
\usepackage[left=2cm,top=2cm,right=2cm,bottom=2cm,nohead,foot=2cm]{geometry}
\geometry{letterpaper}
\usepackage[parfill]{parskip}
\usepackage{graphicx}
\usepackage{amssymb}
\usepackage{amsmath}
\usepackage{float}
\usepackage{epstopdf}
\usepackage{moreverb}
\usepackage{multicol}
\usepackage{comment}
\usepackage{wrapfig}
\DeclareGraphicsRule{.tif}{png}{.png}{`convert #1 `dirname #1`/`basename #1 .tif`.png}
\DeclareGraphicsExtensions{.pdf,.png,.jpg}

%creating solution environment
\specialcomment{sol}{\textbf{Solution: }}{}
%this command toggles the solutions
\excludecomment{sol} %comment this to SHOW solutions

%For the lazy:
\newcommand{\ds}{\displaystyle}
\newcommand{\be}{\begin{enumerate}}
\newcommand{\ee}{\end{enumerate}}
\newcommand{\mrm}{\mathrm}
\newcommand{\bee}{\begin{eqnarray*}}
\newcommand{\eee}{\end{eqnarray*}}
\newcommand{\dds}[2]{\frac{\mathrm{d}#2}{\mathrm{d}#1}} %(d1 BY d2)
\newcommand{\ddb}[2]{\frac{\mathrm{d}}{\mathrm{d}#1}\left[ #2\right]}
%(d by d1 OF 2)

\begin{document}

\begin{minipage}{0.5\textwidth}
		\noindent {\bf CSCI 2824 -- Spring 2020}
\end{minipage}\hfill
\begin{minipage}{0.5\textwidth}
		\noindent \hfill {\bf Homework 6}
\end{minipage}
\noindent This assignment is due on Friday, February 28 to Gradescope by 11:59pm.  You are expected to write up your solutions neatly and \textbf{use the coverpage}.  Remember that you are encouraged to discuss problems with your classmates, but you must work and write your solutions on your own. 

{\bf Important}: On the {\bf official CSCI 2824 cover page} of your assignment clearly write your full name, the lecture section you belong to (001 or 002), and your student ID number.  You may \textbf{neatly} type your solutions for +2 extra credit on the assignment. You will lose \textit{all} 5 style/neatness points if you fail to use the official cover page.

\vspace{5mm}
\be
%==============================================================================
% Set Properties
%==============================================================================
\item In an in-class example, we observed anecdotally that for non-empty sets $P, Q,$ and $R$, the following holds: $$(P \cap Q) \times R = \left( P \times R\right) \cap \left( Q \times R\right)  $$
	\be
		\item Prove that $(P \cap Q) \times R = \left( P \times R\right) \cap \left( Q \times R\right)  $ holds by showing that each side of this equation is a subset of the other side of the equation.
		\item Prove that that $(P \cap Q) \times R = \left( P \times R\right) \cap \left( Q \times R\right)  $ holds using \textbf{set builder} notation and using set identities.
	\ee

	\begin{sol}
		%%YOUR SOLUTION TO #1, HERE!
	\end{sol}

%==============================================================================
% Recurrences
%==============================================================================
\item Complete the following recurrence exercises:
	\be
	\item Find a closed form of the recurrence given by $a_n=7\cdot  a_{n-1}-5;\, a_0=4$
	\item Find a closed form of the recurrence given by $a_n=(n+1)^2\cdot  a_{n-1};\, a_0=1$
	\item Consider the recurrence $a_n=a_{n-1}+2a_{n-2}+2n-9.$
		\be
		\item Show that this recurrence is solved by $a_n=2-n$.
		\item Show that this recurrence is solved by $a_n=2-n+b\cdot2^n$ for any real $b$.
		\ee
	\ee

	\begin{sol}
	%%YOUR SOLUTION TO #2, HERE!
	\end{sol}

%==============================================================================
% Functions: 1-1, Onto, etc.
%==============================================================================
\item Consider the function $f: \mathbb{Z} \times \mathbb{Z} \rightarrow \mathbb{Z}$ given by $f(n,m)=\frac{n^3}{|m|+1}$
	\be
	\item Write in logical predicate notation - using quantifiers for $n$, $m$, etc. as appropriate - the logical equivalent of the function $f$ being onto.
	\item Determine whether or not $f$ is one-to-one.
	\item Determine whether or not $f$ is onto.
	\ee
	\begin{sol}
		%%YOUR SOLUTION TO #3, HERE!
	\end{sol}

%==============================================================================
% Onto and 1-1, as sets
%==============================================================================
\item Define the set $C=$ the set of all Spring 2020 CSCI2824 students.
\be
\item Define in words a function $f: C \rightarrow \mathbb{Z}$.  Is the function function one-to-one and/or onto? Be sure that $f$ is actually a function, and feel free to be creative!
\item Again define the set $C=$ the set of all Spring 2020 CSCI2824 students.  Define in words another function $g: C \rightarrow \mathbb{Z}$, but ensure that $g$ is definitely one-to-one. 
\ee

\begin{sol}
	%%YOUR SOLUTION TO #4, HERE!
\end{sol}

%==============================================================================
% Alg complexity intro
%==============================================================================	
\item The given code attempts to answer the popular debate: ``which Jedi wins in a duel?"  \\
	%\includegraphics[width=.6\textwidth]{JediCode}\\
	The code:
	\be
	\item[i)] Makes sure the inputs - Jedi names, lightsaber colors, and current topographic information - are the same sizes.
	\item[ii)] Pairs up Jedi within a couple of for loops
	\item[iii)] Checks if Jedi \#1 is Mace or if Jedi \#2 is Mace. He always wins in these debates.
	\item[iv)] Checks which Jedi is currently at the highest elevation.  After all, the high ground wins!
	\ee
	You, aspiring Jedi enthusiast, must answer:
	\be
	\item What is the \textbf{algorithmic complexity} of this code?  In other words, exactly how comparisons are checked if $n$ Jedi are input correctly?  You may assume that \textit{each} part of the \textbf{if, elif, else} statement does in fact generate a comparison regardless of input, as this is a ``worst-case" analysis.
	\item What are some redundancies of this code?  Could it be done in less comparisons?
	\item (Not for points): In your opinion, what should actually have been used to determine the winners?
	\ee
	%Sample output here:\\
	%\includegraphics[width=.6\textwidth]{JediOutput}
\begin{sol}
	%%YOUR SOLUTION TO #6, HERE!
\end{sol}

\ee

\end{document}