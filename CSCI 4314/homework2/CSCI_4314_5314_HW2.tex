\documentclass[11pt, oneside]{article}   	% use "amsart" instead of "article" for AMSLaTeX format
\usepackage{geometry}                		% See geometry.pdf to learn the layout options. There are lots.
\geometry{letterpaper}                   		% ... or a4paper or a5paper or ... 
%\geometry{landscape}                		% Activate for for rotated page geometry
%\usepackage[parfill]{parskip}    		% Activate to begin paragraphs with an empty line rather than an indent
\usepackage{graphicx}				% Use pdf, png, jpg, or eps� with pdflatex; use eps in DVI mode
								% TeX will automatically convert eps --> pdf in pdflatex		
\usepackage{amssymb}
\usepackage{amsmath}

\usepackage{enumerate}
\usepackage{booktabs}


\title{CSCI 4314/5314 Dynamic Models in Biology\\Homework Set 2}
\date{} % Activate to display a given date or no date

\begin{document}
\maketitle 

\section*{Simulation of Random Walks (50 points)}

This homework will illustrate a simple example of animal movement predictions through a random walk simulation. In the simulation, you will implement and asses the performance of several of movement schemes in two dimensions: 

\begin{itemize}
\item Biased Random Walks (BRW)
\item Correlated Random Walks (CRW)
\item Biased Correlated Random Walks (BCRW)
\end{itemize}

At each random walk step the components of movement in each direction are given by a weighted vector sum of a correlated term and a biased term:  

\begin{equation}
\Delta X _{i+1} = v\left[ w \cos \left(\theta_0 + \theta^{BRW}_i\right)  +   (1-w) \cos \left(\theta_i + \theta^{CRW}_i \right)\right]
\end{equation}

\begin{equation}
\Delta Y _{i+1} = v \left[ w \sin \left(\theta_0 + \theta^{BRW}_i\right)  +   (1-w) \sin \left(\theta_i + \theta^{CRW}_i \right)\right]
\end{equation}

where $v$ is the step size, $\theta_0$ is the directional bias, $\theta^{CRW}_i$ is an random angel from a uniform probability distribution function with mean 0 and variance $\theta^{*CRW}$.  Similarly, $\theta^{BRW}_i$ is an random angel from a uniform probability distribution function with mean 0 and variance $\theta^{*BRW}$. $w \in [0,1]$ is the weighting given to the directional bias (and hence $(1-w)$ is the weighting given to correlated motion). 

We define the navigational efficiency of a single step of the movement process as:

\begin{equation}
\text{Navigational efficiency} = \frac{\text{Net distance moved towards target in the } \theta_0 \text{ direction}}{\text{Total distance moved}}
\end{equation}


\clearpage

Template code can be found on Canvas under Files/Codes. Feel free to modify the code in any way you prefer: \begin{itemize}
\item Matlab version, file name random\_walks.m
\item Python version, file name random\_walks.py



\end{itemize}

\begin{enumerate}[(1)]  

\item Assume that the variances of the turning angle distributions for the biased and correlated random walk are equal $\theta^{*BRW} = \theta^{*CRW} = \theta^{*}$. Explore several values of turning angle distributions variances $\theta^* = \pi/24,\pi/12,\pi/3$. For each of the following conditions, calculate the mean navigational efficiency (Eq.~3), as a function of simulation time steps, of $50$ repetitions of random walks, each of $500$ time steps:

\begin{enumerate}[(a)]  

\item The case of a pure Biased Random Walks (BRW), i.e., $w=1$. (10 points) 

\item The case of a pure Correlated Random Walks (CRW), i.e., $w=0$. (10 points) 

\item The case of equally balanced BRW and CRW, i.e., $w=0.5$. (10 points) 

\end{enumerate}

\item If $\theta^{*CRW} =\pi/30$, and $\theta^{*BRW} =\pi/3$, what is the value of $w$ associated with the highest navigational efficiency? (20 points)

\end{enumerate}

\section*{Paper Review (50 points)}

The paper ``Evaluating random search strategies in three mammals from distinct feeding guilds'' by Auger--Methe \textit{et. al} (Journal of Animal Ecology, 2016), utilizes several random walk models we learned in class, to classify animal search strategies. You can find the paper on Canvas under Files/Paper Reviews/J Anim Ecol 2016 Auger-Methe.pdf. 

For this part of the homework set, download the paper from Canvas, read it carefully and write a couple of sentences on each of the following items: 

\begin{itemize}
\item What do you feel the main contribution of this paper is? (10 points)
\item What's the essential principle that the paper exploits? (10 points)
\item Describe one major strength of the paper. (10 points)
\item Describe weakness of the paper. (10 points)
\item Describe one future work direction you think should be followed. (10 points)
\end{itemize}


The point of the reviews is demonstrate your understanding of the paper. It is not to regurgitate the paper but to identify what  \textit{you think} is the key concept to learn from the paper and what your opinion is of the strength/weakness of the idea and or paper. We are looking for thoughtful and insightful reviews, that demonstrate depth in your reading and thinking about the paper.


%\clearpage
%
%\textbf{\Large Problem Set Solutions}
%\\
%\\
%
%\begin{enumerate}[(1)] 
%
%\item Histogram of pixel values of f(x, y):
%
%\begin{figure}[h!]
%  \includegraphics[width=0.45\linewidth]{histogram.jpg}
%  %\caption{A boat.}
%  %\label{fig:boat1}
%\end{figure}
%
%
%\item With threshold value $c=1$:
%
%\begin{equation*}
%f(x, y) = \begin{bmatrix} 
%1 & 1 & 1 & 1 \\
%1 & 1 & 1 & 1 \\
%1 & 1 & 1 & 1 \\
%1 & 1 & 1 & 1 
%\end{bmatrix}
%\end{equation*}
%
%\item With threshold value $c=2$:
%
%\begin{equation*}
%f(x, y) = \begin{bmatrix} 
%0 & 1 & 0 & 1 \\
%1 & 1 & 0 & 1 \\
%0 & 1 & 1 & 1 \\
%1 & 0 & 1 & 0 
%\end{bmatrix}
%\end{equation*}
%
%
%With threshold value $c=5$:
%
%\begin{equation*}
%f(x, y) = \begin{bmatrix} 
%0 & 0 & 0 & 0 \\
%0 & 1 & 0 & 0 \\
%0 & 0 & 0 & 0 \\
%0 & 0 & 0 & 0 
%\end{bmatrix}
%\end{equation*}
%
%
%\item 
%
%\begin{equation*}
%f(x, y) = \begin{bmatrix} 
%3 & 2.75  & 1.75 & 1.25\\
%2.4 & 3 & 2.25 & 1.25 \\
%1.75 & 2.5 & 2.5 & 0.75 \\
%1 & 1.25 & 1.25 & 0.25 
%\end{bmatrix}
%\end{equation*}
%
%
%\item 
%
%\begin{equation*}
%k(x, y) = \frac{1}{9} \begin{bmatrix} 
%1 & 1 & 1\\
%1 & 1 & 1\\
%1 & 1 & 1\
%\end{bmatrix}
%\end{equation*}
%
%\item 
%
%\begin{equation*}
%f(x, y) = \begin{bmatrix} 
%1 & 9 & 1 & 4 \\
%4 & 36 & 1 & 9 \\
%1 & 4 & 9 & 4 \\
%9 & 1 & 16 & 1 
%\end{bmatrix}
%\end{equation*}
%
%%\item Rescale the range of pixel values of g(x, y) so that they can be represented by the range \{0,1, 2,...,7\}. 
%
%
%\end{enumerate}
%

%%%%%%%%%%%%%%%%%%%%%%%%%%%%%%%%%%%%%%%%%%%%%%
\end{document}  
















